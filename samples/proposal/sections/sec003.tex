\section{Discussion}
\label{sec:sec003}

To conclude, in this document, I outline some main emerging opportunities, from the literature.
As PAM moves beyond proof-of-concept towards animal monitoring applications, it outcomes the feasibility of this approach into livestock productions.

The low-cost sensors, like \textit{AudioMoth}, have pushed the bottleneck into the analysis and research stages, where it increasingly requires community efforts.
A novel integration between livestock productions and research communities would benefit from it.
Not only concerning reducing costs but also, concerning scientific contributions and research opportunities.

The development of \textit{AudioMoth} is, in short, driving demand from the environmental monitoring and conservation community.
However, it has an applicability range.
From here, I propose the research and development of this device from the livestock domain.
This future work aims at providing the community an extensive survey about livestock populations to investigate patterns and behavior on animal productions.
A future project like this can promote a huge amount of data, which will contribute to further analyses of long-term livestock population trends to aid in future production efforts.

Emerging networked sensors like this, and onboard analysis pipelines, raise the possibility of using PAM data for livestock monitoring.
Deriving detections from sensor networks can provide high spatial and temporal details on data for the livestock activity.
This real-time data can, for instance, be applied to count the number of animals on a herd.
From here, the livestock productions can surpass the need of counting animals manually.
Beyond several barriers, we still face substantial technical difficulties.
A particular difficulty is related to the research retardment of the domain.
Nonetheless, these prospects represent future work to develop technology that is providing sensitive insights into the effects of improving livestock productions.

In short, purchasing available opportunities to apply novel monitoring techniques can dramatically reduce the financial cost and time commitment required for animal production.
Monitoring projects, like the one proposed here, can address more significant questions with access to smart, yet powerful devices, such as \textit{AudioMoth}.
With further developments in the new technologies described here, I am sure that we all can get closer, achieving the basic requirements of a more sustainable animal monitoring, by improving it, as well as improving livestock management.