\section{Background}
\label{sec:sec002}

Medical imaging systems allow the end-user to diagnose several modalities\footnotemark[1], such as MG, US or MRI, from a seamless retrieval medical imaging data~\cite{faraji2019radiologic}.
By bringing those modalities together, it offers new possibilities for quantitative imaging and diagnosis but also requires specialized data handling, post-processing and novel visualization methods~\cite{Igarashi:2016:IVS:2984511.2984537}.
In the clinical domain, medical imaging tools can help experts make better decisions, {\it e.g.}, by identifying cancer prognostics among the available multi-modal data~\cite{Lopes:2017:UHC:3143820.3144118}.

In this document, we focus on understanding different aspects and expectations of a medical imaging Clinical Decision Support Systems (CDSS) integrated into the radiology workflow.
In particular, our work demonstrates how an interactive AI assistant can directly address the above mentioned issues during medical imaging diagnosis.
The following sections discuss recent advances on both CDSS and Human-AI Interaction approaches.

\subsection{Clinical Decision Support Systems}
\label{sec:sec00201}

Most of the best performing CDSS rely on Machine Learning (ML) algorithms that learn specific tasks from training data~\cite{calisto2020breastscreening}.
The field recently gained enormous interest, mostly due to the practical successes of DL~\cite{10.1007/978-3-030-22871-2_67}.
The rapid and widespread development of DL methods supports a wide range of image analysis tasks, including classification, detection, and segmentation \cite{lecun2015deep}.
These methods rely on large annotated data sets to learn essential and discriminative image features for each specific task, with performances matching and even surpassing humans \cite{esteva2017dermatologist}.

In medical applications, deep learning has also been the major contributor to the success of CDSS \cite{esteva2019guide}, \textit{e.g.}, on the diagnosis of skin cancer \cite{esteva2017dermatologist}, the segmentation of cardiac MRI \cite{8759179}, or breast cancer detection \cite{MAICAS2019101562}.
Their outstanding performance in identifying meaningful patterns within the available data was recently used to help humans learn new biomarkers of specific diseases \cite{ wang2019deep}, suggesting these models can see beyond what a trained radiologist sees in medical images.

\subsection{Human-AI Interaction}
\label{sec:sec00202}

Although the research in interaction with intelligent is recent~\cite{burr2018analysis}, still this topic has seen new advances {\it e.g.}, chat-bots and other agents~\cite{miller2019intrinsically}.
Recent advances in medical technologies that promote the generation of data, have continued to drive interaction research in the clinical domain~\cite{azuaje2019artificial}.
Moreover, the new interest of the medical community to support AI research projects and the available public {\it datasets}, are encouraging researchers to work on both fields~\cite{lau2018dataset}.

Human-AI Interaction (HAII) incorporates human feedback in the model training process to create better ML models.
The topic is also known as Interactive Machine Learning (iML)~\cite{10.1145/604045.604056} or Human-In-The-Loop (HITL)~\cite{holzinger2016interactive}.
In this document, we refer to the topic as HAII, that somehow is addressed in~\cite{10.1145/3290605.3300233} providing a set of design guidelines~\cite{10.1145/3132272.3134111}.
In~\cite{Kocielnik:2019:YAI:3290605.3300641} it is also addressed the study of the impact of several methods of expectation setting, and others studied the design for specific HAII scenarios~\cite{aha2017ai}.
While much of the mentioned prior work has employed handcrafted features~\cite{10.1145/3290605.3300233, Kocielnik:2019:YAI:3290605.3300641}, we leverage the rich image data features automatically learned from DL algorithms.

\break