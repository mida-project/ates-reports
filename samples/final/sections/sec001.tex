\section{Introduction}
\label{sec:sec001}

Overstated expectations about assistant agents~\cite{https://doi.org/10.13140/rg.2.2.25412.68486, https://doi.org/10.13140/rg.2.2.33421.59360} and unknown levels of assertiveness decrease user acceptance and adoption of clinical systems, in particular when those expectations are not met~\cite{Kocielnik:2019:YAI:3290605.3300641}.
The recent hype of Artificial Intelligence (AI) techniques brings a new paradigm in the context of Human-AI interaction.
Modern {\it AI-Assisted} technologies, such as Deep Learning (DL) methods, open new possibilities in the clinical domain~\cite{topol2019high}, including:
(i) genome interpretation~\cite{sundaram2018predicting};
(ii) medical coaching via a smart speaker ({\it e.g.}, Alexa)~\cite{bickmore2018patient};
(iii) assistive scan readings~\cite{madani2018deep};
(iv) cancer diagnosis and identification of mutations~\cite{coudray2018classification}; and
(v) mortality prediction~\cite{ahmad2018death}.
Although, these methods exhibit good performance accuracy, most physicians do not expect their clinical systems to behave inconsistently and imperfectly~\cite{Kocielnik:2019:YAI:3290605.3300641}, these assistant agents are programmed to follow specific predefined diagnostic that often are equal for all clinicians.
In most cases, assistants do not consider the clinician's professional years of medical experience.

% ({\it i.e.}, Interns, Juniors, Middles and Seniors)

Having mechanisms to obtain and analyse clinicians' behaviour, can help researchers to create a more reliable assistant agent~\cite{Miao2019}.
Concerning medical experience and clinical profile ({\it e.g.}, radiologists, surgeons, etc), we will analyse critical behaviour and implement persuasive mechanisms to reduce the rates of FPs and FNs.
In particular, personalized persuasion strategies are more effective than non-personalized strategies in reporting the diagnostic or influencing the clinician to take the right decision~\cite{sonntag2016persuasive}.
Thus, it will add great value for researchers in the development of assistant agents for the diagnostic via medical imaging.

\clearpage

Recently, deep learning methods started outperforming traditional machine learning algorithms in many application domains, where the access to large amount of data and training {\it datasets} was feasible~\cite{mckinney2020international}.
The work described here is one such example.
Specifically, we introduce a  DL methodology in multi-modal ({\it Multimodality}) breast imaging\footnotemark[1] system~\cite{calisto2017mimbcdui, https://doi.org/10.13140/rg.2.2.29816.70409}.
To this end, we will develop {\it BreastScreening-AI}, a multimodal {\it AI-Assisted} medical imaging framework that allows radiologists to visualize and manipulate images, while also accessing a diagnostic recommendation from an assistant agent.
The {\it BreastScreening} framework is used to test the usefulness of an intelligent medical that is integrated in an user interface.
We have collected a large breast imaging database. This database contains 338 multi-modal images cases. Specifically, it contains mammography, ultrasound and magnetic resonance imaging.
Also, our database contains the manual delineation of the lesion ({\it i.e.}, ground-truth) along with the lesion severity, {\it i.e.}, Breast Imaging Reporting and Data System~\textemdash~BI-RADS~\cite{ghosh2019artificial}.

%%%%%%%%%%%%%%%%%%%%%%%%%%%%%%%%%%%%%%%%%%%%%%%%%%%
\footnotetext[1]{\scriptsize {\it Multi-Modality}: diagnostic technique for the patient treatment via: (1) MammoGraphy (MG), both CranioCaudal (CC) and MedioLateral Oblique (MLO) views; (2) UltraSound (US); (3) Magnetic Resonance Imaging (MRI); and (4) text. The considered text modalities are, for instance, report information, personal history, family history, age, among others. {\it BreastScreening}: stands for the general framework and the developed platform. {\it BreastScreening-AI}: stands for the integration of our {\it AI-Assisted} methods with a visual assistant agent, into the {\it BreastScreening} platform. The goal of the assistant is to interact with the user in two scenarios: (1) Assertive; and (2) Non-Assertive.}
%%%%%%%%%%%%%%%%%%%%%%%%%%%%%%%%%%%%%%%%%%%%%%%%%%%

Here, we present a proposal for applying {\it BreastScreening-AI} in two scenarios, where clinicians will interact with Assertive and Non-Assertive assistant agents~\cite{pacheco2019alignment, 10.1145/3311350.3347162}.
The assistant will act as a second reader, providing results in terms of improvements on the classification diagnosis ({\it i.e.}, Over-Diagnosis {\it vs} Under-Diagnosis) regarding FPs and FNs, as well as efficiency and efficacy in the workflow.
While considerable work as focused on improving the accuracy of AI algorithms, comparatively less work focused on improving adoption and usability of interactive assistance techniques.
This proposal contributes broadly to our work by examining what clinicians need when using AI-powered image diagnostic, the practices they adopt while using diagnostic tools, and how these diagnostic tools affect the end-user attitudes towards the underlying AI algorithms.