\section{Analysis}
\label{sec:sec003}

For this document, our goal is to propose a new study.
The study will describe how displaying different levels of assertiveness can influence clinician's responses to agents.
Moreover, we expect to understand the clinicians' behaviour during the decision-making process in a real-world setting.
In this section, we propose the analysis of a study where we devise a mixed-design prototype in which we will manipulate the level of assertiveness displayed by the assistant.

\subsection{Medical Procedures}
\label{sec:sec00301}

At this point, we will take impressions regarding the efficiency of clinicians, and their recommendations based on their experience for improvements of the patient examination.
In fact, several studies demonstrated~\cite{waite2017tired} that radiologist fatigue levels and performance are related to environmental factors such as number of FPs and FNs.
That said, we start analyzing the potential enhancement that an {\it AI-Assisted} diagnosis could take in the radiology room~\cite{chatelain2018evaluation, miglioretti2007radiologist}.

\subsection{Insights and Challenges}
\label{sec:sec00302}

Our observations and interviews will align with previous research on clinician-driven diagnostic {\it tasks}~\cite{Sultanum:2018:MTP:3173574.3173996}.
From the research insights we need to identify the following main challenges:
i) the heterogeneous visualization mode of a large number of images and file sizes;
ii) the annotation of medical images to support diagnosis and also 
how the introduction of the {\it AI techniques} can improve the classification ground-truth for; and
iii) when performing the classifications, the clinicians' gap in visualizing images from different modalities.

\subsection{Design Goals}
\label{sec:sec00303}

As demonstrative example of implementing a diagnostic assistant in the design of medical imaging systems, we need to propose several design goals.
The main design goals should be closely related to the research insights and the challenges of the previous section, namely:
(1) a collection of a ground truth annotations, namely masses in all imaging modalities and calcification lesions in MG (for both CC and MLO views);
(2) classification of the lesion severity using the BI-RADS~\cite{aghaei2018association};
(3) categorization of the breast tissues (dense vs non-dense);
(4) clinical co-variables, such as personal and family records; and
(5) visualizations for clinical summary which is crucial for a proper diagnosis and to perform patient follow-up.

\subsection{Design Methods}
\label{sec:sec00304}

For the future study, we will need to actively involve all clinicians in the design of this medical imaging solution.
To generate clinician's empathy and involvement, design methods from participatory design will be considered~\cite{10.1145/3025453.3025873}.

\break

\noindent
Our design methods should consist of three aspects:

\begin{itemize}
\begin{minipage}{0.3\linewidth}
\item {\it insight}
\end{minipage}
\begin{minipage}{0.3\linewidth}
\item {\it ideation}
\end{minipage}
\begin{minipage}{0.3\linewidth}
\item {\it implementation}
\end{minipage}
\end{itemize}

\hfill

In analysis of clinicians' needs and requirements, we will focus on how the aforementioned aspects of our design methods are interpreted, achieved or disregarded by clinicians.
For that, in the final report we will explain the details of each aspect concerning the proposed design methods.

\subsection{Research Questions}
\label{sec:sec00305}

The accuracy level of a clinical system is defined as the total number of correct predictions over all possible predictions~\cite{seref2019performance}.
This definition requires the use of the following error metrics:
(1) FP; and
(2) FN.
Typically, FP and FN are used to quantify {\bf Precision} {\it vs} {\bf Recall}.
In general, clinical systems are optimized for high precision and, therefore, avoid FPs ({\it i.e.}, in our context, avoid recommending a BI-RADS higher than the real one).

Previous works outside of the clinical scope~\cite{Kocielnik:2019:YAI:3290605.3300641, Dove:2017:UDI:3025453.3025739}, denote that the impact of FP {\it vs.} FN on UX is generally unexplored.
However, thus is of high relevance when considering AI systems for the clinical domain~\cite{boughey2016identification, dialani2015role} as it will be experimentally shown.

We also want to measure that our \textit{AI-Assistant} as a function in two scenarios ({\it i.e.}, Assertive and Non-Assertive) of the above metrics.
Measuring predictions are typically quantified as precision in contrast with recall.
We, therefore, explore the following {\it Research Questions} and associated each to the set of {\it Hypothesis} following the guidelines described in~\cite{10.1145/3290605.3300233, Kocielnik:2019:YAI:3290605.3300641}.
Specifically, we consider the following research questions and related hypothesis:

%%%%%%%%%%%%%%%%%%%%%%%%%%%%%%%%%%%%%%%%%%%%%%%%%%%
\begin{itemize}
\item {\bf RQ1.} Should the assistant agent interact with all clinicians in the same way?
\begin{itemize}
\item {\bf H1.1.} Less experienced clinicians are performing better with more assertive assistants.
\item {\bf H1.2.} Higher experienced clinicians are performing better with low assertive assistants.
\end{itemize}
\item {\bf RQ2.} Will the assertiveness perception of the assistant agent improve UX?
\begin{itemize}
\item {\bf H2.1.} The UX of assertive assistant agent is better for less experienced clinicians.
\item {\bf H2.2.} The UX of non-assertive assistant agent is better for higher experienced clinicians.
\end{itemize}
\end{itemize}
%%%%%%%%%%%%%%%%%%%%%%%%%%%%%%%%%%%%%%%%%%%%%%%%%%%

The work in~\cite{10.1145/3290605.3300233}, describes a set of \underline{18 guidelines} for Human-AI Interaction (HAII) being highly useful to answer {\bf RQ2.} question.
Also, in~\cite{Kocielnik:2019:YAI:3290605.3300641} it is provided  an exploratory study of an {\it AI-Assistant} to study the impact of several methods of expectation-setting, also answering {\bf RQ2.} question.
In both studies, the authors show that different focus on avoiding types of errors lead to a quite different subjective perceptions ({\it i.e.}, the {\bf RQ1.} question) of accuracy and acceptance.

\break