\documentclass[sigchi]{acmart}

\usepackage[utf8]{inputenc}
\usepackage[english]{babel}
\usepackage{booktabs} 	% For formal tables
\usepackage{listings}   % Writing code in latex document
\usepackage{hyperref} 	% For hyperlinks
\usepackage{appendix}
\usepackage{chngcntr}

\hypersetup{
  colorlinks=true,
  linkcolor=blue,
  filecolor=blue,      
  urlcolor=blue,
}

\graphicspath{{images/}}

\begin{document}
\title[Final Project]{\normalsize Final Project: Advanced Topics in Entertainment Systems}
\subtitle{Impact of Assertiveness-based Interaction in Medical Imaging Diagnosis}

\author{Francisco Maria Calisto}
\orcid{0000-0001-8179-7872}
\affiliation{%
  \institution{Instituto Superior T\'{e}cnico}
  \city{Lisboa}
  \state{Portugal}
  \postcode{1049-001}
}
\email{francisco.calisto@tecnico.ulisboa.pt}

% The default list of authors is too long for headers.
\renewcommand{\shortauthors}{Calisto}


\begin{abstract}

This report provides information concerning the study for the \textit{Impact of Assertiveness-based Interaction in Medical Imaging Diagnosis}.
In context of this document, it was created for the \hyperlink{https://fenix.tecnico.ulisboa.pt/disciplinas/TASE4/2019-2020/2-semestre}{Advanced Topics in Entertainment Systems (ATES)} course of the \hyperlink{https://fenix.tecnico.ulisboa.pt/cursos/deic/curriculo}{Computer Science and Engineering Doctoral Program (PDEI)} at \hyperlink{https://tecnico.ulisboa.pt/en/}{Instituto Superior T\'{e}cnico (IST)} from \hyperlink{https://www.ulisboa.pt/}{University of Lisbon (ULisboa)} - \hyperlink{https://www.portugal.gov.pt/en/}{Portugal} (\hyperlink{https://europa.eu}{EU}).

This document describes the final project of a study to understand how the level of assertiveness, displayed by an assistant agent, will impact the radiologists' decision-making process during breast cancer diagnosis.
We will develop a proof-of-concept prototype, a medical imaging assistant, with two major scenarios of the assistant behaviour:
(1) one scenario for an Assertive assistant; and
(2) a second scenario for a Non-Assertive assistant;
paired with two assistant behaviours:
(i) Proactive; and
(ii) Reactive.
We will conduct a 4x4, within-subjects experiment (n = 10), in which the four Assertive, Non-Assertive, Proactive and Reactive subjects will be matched with the four categories of radiologists' professional experience, {\it i.e.}, Interns, Juniors, Middles and Seniors.
From a latter study, we will extract the rates of False-Positives (FPs) and False-Negatives (FNs) to understand patterns of critical behaviour among the four categories of professional experience.
In the end, we will propose several strategies that we expect to promote the unbiased behaviour per each category of professional experience, improving the FPs and FNs rates during diagnostic.

\end{abstract}

\begin{teaserfigure}
\includegraphics[width=\textwidth]{teaser}
\end{teaserfigure}

\maketitle

\section{Introduction}
\label{sec:sec001}

~\cite{https://doi.org/10.13140/rg.2.2.25412.68486, https://doi.org/10.13140/rg.2.2.33421.59360}

~\cite{https://doi.org/10.13140/rg.2.2.29816.70409}

~\cite{calisto2017mimbcdui}

\break
\section{Background}
\label{sec:sec002}

Medical imaging systems allow the end-user to diagnose several modalities\footnotemark[1], such as MG, US or MRI, from a seamless retrieval medical imaging data~\cite{faraji2019radiologic}.
By bringing those modalities together, it offers new possibilities for quantitative imaging and diagnosis but also requires specialized data handling, post-processing and novel visualization methods~\cite{Igarashi:2016:IVS:2984511.2984537}.
In the clinical domain, medical imaging tools can help experts make better decisions, {\it e.g.}, by identifying cancer prognostics among the available multi-modal data~\cite{Lopes:2017:UHC:3143820.3144118}.

In this document, we focus on understanding different aspects and expectations of a medical imaging Clinical Decision Support Systems (CDSS) integrated into the radiology workflow.
In particular, our work demonstrates how an assertiveness-based interaction can improve the medical imaging diagnosis.
The following sections discuss recent advances on both CDSS and Human-AI Interaction approaches.

\subsection{Clinical Decision Support Systems}
\label{sec:sec00201}

Most of the best performing CDSS rely on Machine Learning (ML) algorithms that learn specific tasks from training data~\cite{calisto2020breastscreening}.
The field recently gained enormous interest, mostly due to the practical successes of DL~\cite{10.1007/978-3-030-22871-2_67}.
The rapid and widespread development of DL methods supports a wide range of image analysis tasks, including classification, detection, and segmentation \cite{lecun2015deep}.
These methods rely on large annotated data sets to learn essential and discriminative image features for each specific task, with performances matching and even surpassing humans \cite{esteva2017dermatologist}.

In medical applications, deep learning has also been the major contributor to the success of CDSS \cite{esteva2019guide}, \textit{e.g.}, on the diagnosis of skin cancer \cite{esteva2017dermatologist}, the segmentation of cardiac MRI \cite{8759179}, or breast cancer detection \cite{MAICAS2019101562}.
Their outstanding performance in identifying meaningful patterns within the available data was recently used to help humans learn new biomarkers of specific diseases \cite{ wang2019deep}, suggesting these models can see beyond what a trained radiologist sees in medical images.

\subsection{Human-AI Interaction}
\label{sec:sec00202}

Although the research in interaction with intelligent is recent~\cite{burr2018analysis}, still this topic has seen new advances {\it e.g.}, chat-bots and other agents~\cite{miller2019intrinsically}.
Recent advances in medical technologies that promote the generation of data, have continued to drive interaction research in the clinical domain~\cite{azuaje2019artificial}.
Moreover, the new interest of the medical community to support AI research projects and the available public {\it datasets}, are encouraging researchers to work on both fields~\cite{lau2018dataset}.

Human-AI Interaction (HAII) incorporates human feedback in the model training process to create better ML models.
The topic is also known as Interactive Machine Learning (iML)~\cite{10.1145/604045.604056} or Human-In-The-Loop (HITL)~\cite{holzinger2016interactive}.
In this document, we refer to the topic as HAII, that somehow is addressed in~\cite{10.1145/3290605.3300233} providing a set of design guidelines~\cite{10.1145/3132272.3134111}.
In~\cite{Kocielnik:2019:YAI:3290605.3300641} it is also addressed the study of the impact of several methods of expectation setting, and others studied the design for specific HAII scenarios~\cite{aha2017ai}.
While much of the mentioned prior work has employed handcrafted features~\cite{10.1145/3290605.3300233, Kocielnik:2019:YAI:3290605.3300641}, we leverage the rich image data features automatically learned from DL algorithms.

\break
\section{Clinical Design Keys}
\label{sec:sec003}

For this document, our goal is to propose a new study.
The study will describe how displaying different levels of assertiveness can influence clinician's responses to agents.
Moreover, we expect to understand the clinicians' behaviour during the decision-making process in a real-world setting.
In this section, we propose the analysis of a study where we devise a mixed-design prototype in which we will manipulate the level of assertiveness displayed by the assistant.

\subsection{Medical Procedures}
\label{sec:sec00301}

At this point, we will take impressions regarding the efficiency of clinicians, and their recommendations based on their experience for improvements of the patient examination.
In fact, several studies demonstrated~\cite{waite2017tired} that radiologist fatigue levels and performance are related to environmental factors, such as number of FPs and FNs.
That said, we start analyzing the potential enhancement, a certain {\it AI-Assisted} diagnosis could take in the radiology room~\cite{chatelain2018evaluation, miglioretti2007radiologist}.

\subsection{Insights and Challenges}
\label{sec:sec00302}

Our observations and interviews will align with previous research on clinician-driven diagnostic {\it tasks}~\cite{Sultanum:2018:MTP:3173574.3173996}.
From the research insights, we need to identify the following main challenges:
i) the heterogeneous visualization mode of a large number of images and file sizes;
ii) the annotation of medical images to support diagnosis and also 
how the introduction of the {\it AI techniques} can improve the classification ground-truth for; and
iii) when performing the classifications, the clinicians' gap in visualizing images from different modalities.

\subsection{Design Goals}
\label{sec:sec00303}

As demonstrative example of implementing a diagnostic assistant in the design of medical imaging systems, we need to propose several design goals.
The main design goals should be closely related to the research insights and the challenges of the previous section, namely:
(1) a collection of a ground truth annotations, namely masses in all imaging modalities and calcification lesions in MG (for both CC and MLO views);
(2) classification of the lesion severity using the BI-RADS~\cite{aghaei2018association};
(3) categorization of the breast tissues (dense vs non-dense);
(4) clinical co-variables, such as personal and family records; and
(5) visualizations for clinical summary which is crucial for a proper diagnosis and to perform patient follow-up.
The aforementioned design goals aim at promoting a reliable diagnosis information to clinical radiologists.
Pairwise with the related literature and our formative study, we found a number of current issues early addressed, including {\it medical imaging structure trade-offs}, {\it radiology room temporal awareness}, {\it image segmentation} and {\it radiologists system trust}.

\noindent
We fuse these five insights into three corresponding design goals, as follows:

%%%%%%%%%%%%%%%%%%%%%%%%%%%%%%%%%%%%%%%%%%%%%%%%%%
\begin{description}
\item[Medical Imaging Design (MID)] focusing on how to provide the best visualization strategy, given the heterogeneous information coming from the multi-modal sources of information;

\item[Control Result Design (CRD)] focusing on improving the physician's ability to {\it accept} or {\it reject} the {\it AI-Assisted} results;

\item[Based Explanation Design (BED)] focusing on increasing physicians understanding of how the AI techniques operate. By increasing understanding of how AI works, physicians can update their expectations of how well and in which situations the system is likely to work;
\end{description}
%%%%%%%%%%%%%%%%%%%%%%%%%%%%%%%%%%%%%%%%%%%%%%%%%%

\subsection{Design Methods}
\label{sec:sec00304}

For the future study, we will need to actively involve all clinicians in the design of this medical imaging solution.
To generate clinician's empathy and involvement, design methods from participatory design will be considered~\cite{10.1145/3025453.3025873}.

\hfill

\noindent
Our design methods should consist of three aspects:

\begin{itemize}
\begin{minipage}{0.3\linewidth}
\item {\it insight}
\end{minipage}
\begin{minipage}{0.3\linewidth}
\item {\it ideation}
\end{minipage}
\begin{minipage}{0.3\linewidth}
\item {\it implementation}
\end{minipage}
\end{itemize}

These aspects, are aligned with the analysis of clinicians' needs and requirements.
More precisely, we will focus on how the aforementioned aspects of our design methods are interpreted, achieved or disregarded by clinicians.

First of all, methodologies are useful to broaden our thinking.
For example, techniques such as interviews and observations are helping us to have a synthesized {\it insight} in the clinical workflow.
According to this aspect, we went through several observations and interviews on clinical institutions.

Secondly, {\it ideation}, the process of generating new ideas, is central to design where the goal is to find novel solutions around a set of user needs and requirements.
Therefore, we promote several brainstorming techniques.
Those techniques are such as individual interviews, focus groups and affinity diagrams.
Affinity diagramming has been used in our study to organize the acquired large sets of ideas into data clusters.
In this document, the methods are used to organize our findings and to sort design ideas into {\it ideation} of a focus group during several meetings.

Third and final, we suggest the need for new interactive design methods to externalise thoughts and ideas, forcing clinicians to be more explicit.
For that purpose, we will promote the {\it implementation} of rapid prototyping solutions.
In developing this prototyping approach, we quickly recognized that successful {\it implementation} would rely on a bare minimum number of requirements.
Short iterations will enable the use of many different design methods for prototyping and testing, as we have many different concerns with clinicians.

\subsection{Research Questions}
\label{sec:sec00305}

The accuracy level of a clinical system is defined as the total number of correct predictions over all possible predictions~\cite{seref2019performance}.
This definition requires the use of the following medical error metrics:
(1) FP; and
(2) FN.
Typically, FP and FN are used to quantify {\bf Precision} {\it vs} {\bf Recall}.
In general, clinical systems are optimized for high precision and, therefore, avoid FPs ({\it i.e.}, in our context, avoid recommending a BI-RADS higher than the real one).

Previous works outside of the clinical scope~\cite{Kocielnik:2019:YAI:3290605.3300641, Dove:2017:UDI:3025453.3025739}, denote that the impact of FP {\it vs.} FN on UX is generally unexplored.
However, thus is of high relevance when considering AI systems for the clinical domain~\cite{boughey2016identification, dialani2015role} as it will be experimentally shown.

We also want to measure that our \textit{AI-Assistant} as a function in two scenarios ({\it i.e.}, Assertive and Non-Assertive) of the above metrics.
Measuring predictions are typically quantified as precision in contrast with recall.
We, therefore, explore the following {\it Research Questions} and associated each to the set of {\it Hypothesis} following the guidelines described in~\cite{10.1145/3290605.3300233, Kocielnik:2019:YAI:3290605.3300641}.

\hfill

\noindent
Specifically, we consider the following research questions and related hypothesis:

%%%%%%%%%%%%%%%%%%%%%%%%%%%%%%%%%%%%%%%%%%%%%%%%%%%
\begin{itemize}
\item {\bf RQ1.} Should the assistant agent interact with all clinicians in the same way to improve their performance?
\begin{itemize}
\item {\bf H1.1.} Less experienced clinicians are performing better with more assertive assistants.
\item {\bf H1.2.} Higher experienced clinicians are performing better with low assertive assistants.
\end{itemize}
\item {\bf RQ2.} Will the perception of assertiveness-based assistance be the same per professional category?
\begin{itemize}
\item {\bf H2.1.} Less experience clinicians will prefer to interact with more assertive agents.
\item {\bf H2.2.} Higher experience clinicians will prefer to interact with less assertive agents.
\end{itemize}
\item {\bf RQ3.} Adapting the assertiveness levels to the professional category will result into UX improvements?
\begin{itemize}
\item {\bf H3.1.} The UX of assertive agent is better for less experienced clinicians.
\item {\bf H3.2.} The UX of non-assertive agent is better for higher experienced clinicians.
\end{itemize}
\end{itemize}
%%%%%%%%%%%%%%%%%%%%%%%%%%%%%%%%%%%%%%%%%%%%%%%%%%%

The work in~\cite{10.1145/3290605.3300233}, describes a set of \underline{18 guidelines} for Human-AI Interaction (HAII) being highly useful to answer both {\bf RQ2.} and {\bf RQ3.} questions.
Also, in~\cite{Kocielnik:2019:YAI:3290605.3300641} it is provided  an exploratory study of an {\it AI-Assistant} to study the impact of several methods of expectation-setting, answering {\bf RQ2.} and {\bf RQ3.} questions.
In both studies, the authors show that different focus on avoiding types of errors lead to a quite different subjective perceptions ({\it i.e.}, the {\bf RQ1.} question) of accuracy and acceptance.

\break


\clearpage

\bibliographystyle{ACM-Reference-Format}
\bibliography{bibliography}

\clearpage

\appendix
\addappheadtotoc
\counterwithin{figure}{section}
\section{Appendix}
\label{sec:sec010}

%%%%%%%%%%%%%%%%%%%%%%%%%%%%%%%%%%%%%%%%%%%%%%%%%%%
\begin{figure}[htbp]
\onecolumn
\centering
\includegraphics[width=0.95\textwidth]{fig022}
\caption{When the clinician press the "Accept" button, the system will trigger two minor scenario options: (a) {\bf Active}; or (b) {\bf Passive}. During our sessions of low-fi prototype co-design with clinicians, all of them preferred the (b) {\bf Passive} option.}
\label{fig:fig022}
\twocolumn
\end{figure}
%%%%%%%%%%%%%%%%%%%%%%%%%%%%%%%%%%%%%%%%%%%%%%%%%%%

%%%%%%%%%%%%%%%%%%%%%%%%%%%%%%%%%%%%%%%%%%%%%%%%%%%
\begin{figure}[htbp]
\onecolumn
\centering
\includegraphics[width=0.95\textwidth]{fig002}
\caption{Our {\it BreastScreening} assistant provides several features regarding the basics of medical imaging diagnostic. From there, we will be able to validate our DenseNet BIRADS classifier along with clinicians.}
\label{fig:fig002}
\twocolumn
\end{figure}
%%%%%%%%%%%%%%%%%%%%%%%%%%%%%%%%%%%%%%%%%%%%%%%%%%%

\clearpage

%%%%%%%%%%%%%%%%%%%%%%%%%%%%%%%%%%%%%%%%%%%%%%%%%%%
\begin{figure}[t!]
\onecolumn
\centering
\includegraphics[width=\textwidth]{fig024}
\caption{Here, we propose the suggested system resulted from a preliminary evaluation. From this proposed system, we will be able to understand when ({\it e.g.}, {\bf Proactive} {\it vs} {\bf Reactive}) and how ({\it e.g.}, {\bf Assertive} {\it vs} {\bf Non-Assertive}) should the assistant agent adapt and change the behaviour per each group ({\it i.e.}, Interns, Juniors, Middles and Seniors) of medical experience. In this case, we have an {\bf Assertive} communication with the second screen of a {\bf Reactive} agent.}
\label{fig:fig024}
\twocolumn
\end{figure}
%%%%%%%%%%%%%%%%%%%%%%%%%%%%%%%%%%%%%%%%%%%%%%%%%%%

\end{document}
