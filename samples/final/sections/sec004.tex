\section{Methods}
\label{sec:sec004}

We will evaluate {\it BreastScreening-AI} simulating real-world conditions with 10 clinicians (expected) from several Portuguese clinical institutions.
Our goal will be to quantitatively and qualitatively assess the proposed design principles that the {\it BreastScreening-AI} system embodies (rather than on particular widgets of the UI) and to understand how these principles would fare in practice.
We will be particularly interested in understanding how {\it BreastScreening-AI} would accomplish improvements of diagnosis, surpassing the challenges and providing evidence of the opportunities emerging from HAII.
Ultimately, we will focus on clinicians' accuracy while using our {\it BreastScreening-AI} features and how to improve AI reliability during medical decision-making.

\subsection{Study Setup}
\label{sec:sec00403}

For the study, we will conduct both quantitative and qualitative analysis.
The quantitative analysis will focus on measuring the clinicians' performance ({\it e.g.}, number of False-Positives and False-Negatives) during diagnosis of three groups of patients ({\it i.e.}, low, medium and high severity) for each doctor.
The above measurements are part of the quantitative analysis with a comparison between Assertive and Non-Assertive scenarios.
With that comparison, we will answer the  research questions and hypothesis mentioned in Section~\ref{sec:sec00305}, providing evidence for the impact, expectations and acceptance of {\it AI-Assistance} on the radiology room workflow.
For the qualitative analysis, we will extract opinion-based feedback from recorded audios.

\subsection{Procedure}
\label{sec:sec00404}

At this stage, each participant will interact with the assistant, {\it accepting} or {\it rejecting} the system suggestion in the two different scenarios.
The set of patients will provide participants with 289 patients, while all patients must have at least one of the three available modalities.
Each participant will open the respective set of three patients ({\it e.g.}, {\bf P1}, {\bf P2} or {\bf P3}), chosen randomly, and will examine the set of images.
During the examination, the participant will interact with the available features of the system.

\subsection{Measures}
\label{sec:sec00405}

We will measure system accuracy through three questions adapted from the Model of Trust~\cite{schoorman2016perspective} that we called Dimensions Of Trust Scale (DOTS).
The three questions will be answered on a 20-point scale with 5\% increments.
Furthermore, we will do several {\it post-task} questions.
The {\it post-task} questions, aim to evaluate satisfaction.
We will also measure the recall and precision ({\it e.g.}, through the FP and FN rates) of the system.

\clearpage